\documentclass{article}

% Language setting
% Replace `english' with e.g. `spanish' to change the document language
\usepackage[brazil]{babel}
\usepackage[utf8]{inputenc}

% Set page size and margins
% Replace `letterpaper' with `a4paper' for UK/EU standard size
\usepackage[a4paper,top=2cm,bottom=2cm,left=3cm,right=3cm,marginparwidth=1.75cm]{geometry}
\usepackage{url}

% Useful packages
\usepackage{amsmath}
\usepackage{graphicx}
\usepackage[colorlinks=true, allcolors=blue]{hyperref}

\title{Pesquisa sobre os benefícios da qualidade de software, melhoria de processos e Melhoria do Processo de Software Brasileiro (MPSBR), com foco no Nível B, em aplicações comerciais de software.}
\author{Maruan Biasi El Achkar, Victor Matheus Moreira, e Ricardo Falcão Schlieper}

\begin{document}
\maketitle

\begin{abstract}
A qualidade de software pode ser alcançada pela implementação de métodos de melhoria de processos, como o MPSBR, neste artigo, exploraremos os detalhes dessa área, simplificando sua complexidade e evidenciando os benefícios e desafios que podem ser esperados.
\end{abstract}

\section{Introdução}

Na tecnologia, garantir a qualidade das aplicações desenvolvidos é de extrema importância, para isso, métodos de verificação e melhoria de processos e produtos foram desenvolvidos, entre eles, o programa MPSBR (Melhoria do Processo de Software Brasileiro) se destaca no Brasil por ser um sistema desenvolvido e aprimorado nacionalmente.

\section{Introdução a qualidade de software}

A área de qualidade de software é extensa e abrangente, porém pode ser dividida em duas categorias principais: qualidade de produto e qualidade de processo.

\subsection{Qualidade de Produto}

 Qualidade de produto são as normas criadas para o produto final, elas definem o objetivo a ser alcançado pelos processos. Um bom exemplo é uma um hambúrguer de frango, podemos dizer que nesse caso as normas são: O hambúrguer deve conter dois pães, alface, tomate, frango, queijo e maionese. Se o produto final tiver todos esses itens, ele será aprovado na qualidade de produto, pois o objetivo foi alcançado.

\subsection{Qualidade de Processo}
Qualidade de processo são as normas criadas para os processos que juntos dão origem ao produto final. Continuando com o exemplo do hambúrguer de frango, algumas normas de processo são: A montagem do hambúrguer deve começar pelo pão de baixo, seguido pelo frango, queijo, alface, tomate, molho e por último o pão de cima. Dessa forma, criamos um checklist que só nos permite passar para a próxima etapa quando a atual estiver concluída, dessa forma, garantirmos que nenhuma das normas do produto, ou seja, do objetivo final, será ignorada durante os processos de montagem desse hambúrguer. Só podemos adicionar o frango caso o pão de baixo já esteja ali, da mesma forma que só podemos colocar os tomates após colocarmos a alface.

\subsection{Conclusão}
Podemos concluir que garantir a qualidade de produto e processo é melhor do que garantir apenas uma, pois as duas se complementam. Precisamos da qualidade de produto para criarmos a qualidade de processo, que irá nos auxiliar a garantir a qualidade de produto. A presença das duas formas de garantia de qualidade facilita a aplicação das duas em um software, elas devem ser consideradas um par, uma precisando da outra para funcionar da melhor forma.

\section{Aplicação}

Após as pesquisas conduzidas, vimos que garantir a qualidade dos processos é mais importante do que garantir a qualidade do produto final, já que bons processos geram uma boa execução. Primeiro, precisamos definir as normas do produto, para isso, utilizemos um novo exemplo, um software de foguete.

\begin{itemize}
  \item O software deve ser parrudo, não permitindo erros de execução por conta de variáveis erradas.
  \item O software não pode permitir que dados gerados por usuários acarretem erros.
\end{itemize}
Para saber se nosso software tenha qualidade, precisamos testar se ele segue as normas, o problema é: caso a aplicação tenha algum problema, precisaremos voltar para a fase de desenvolvimento até que o problema seja resolvido, gastando recursos valiosos como tempo e dinheiro. Uma solução simples e eficaz é a introdução da qualidade de processo, para isso, precisamos de normas de processo.

\begin{itemize}
  \item Os desenvolvedores devem utilizar linguagens não tipadas.
  \item Todos os dados gerados por usuários devem ser tratados quanto antes possível, se não for possível, esses dados devem ser excluídos e o programa precisa conseguir continuar funcionando sem eles.
\end{itemize}
Com essas normas, precisamos apenas garantir que elas estão sendo seguidas para automaticamente garantirmos que nosso produto será de qualidade, pois com elas, evitamos todos os possíveis problemas de produto já na fase de desenvolvimento, economizando recursos. Com esse exemplo, fica fácil de perceber como a qualidade de processo evita problemas futuros.

\section{Melhoria do Processo de Software Brasileiro (MPSBR)}
A Melhoria do Processo de Software Brasileiro, também conhecido pela sua sigla, MPSBR, é um programa desenvolvido pela Organização Social Civil de Interesse Público Softex (Associação para Promoção da Excelência do Software Brasileiro) visando melhorar a capacidade de desenvolvimento de software, serviços e gestão na área da tecnologia. O método teve o início de seu desenvolvimento em dezembro de 2003 com o apoio do Ministério da Ciência, Tecnologia e Inovações (MCTI). \\
O sistema MPSBR é divido em três modelos de referência:

\begin{itemize}
    \item MR-MPS-SW: Processo de Software.
    \item MR-MPS-SV: Processo de Serviços.
    \item MR-MPS-RH: Processo de Gestão de Pessoas.
\end{itemize}
Em janeiro de 2024, mais de 950 empresas já haviam sido avaliadas pelos métodos da Softex, a grande maioria no modelo de referência SW e apenas duas no modelo RH. A lista completa pode ser acessada pelo site softex.br/mpsbr/avaliacoes. \\
Como o foco deste artigo é pesquisar a qualidade de produtos e processos de software, focaremos no MR-MPS-SW, sua base técnica vem do NBR ISO/IEC 12.207, das normas inclusas no ISO/IEC 33001 e do CMMI®.
\subsection{NBR ISO/IEC 12.207}
A NBR ISO/IEC 12.207 especifica uma estrutura e terminologias para os processos inclusos no ciclo de vida de um software. A estrutura define tarefas a serem executadas durante as fases de fornecimento, desenvolvimento, operação, manutenção e descontinuidade de um produto, além da parte comercial de venda, aquisição e implementação do sistema.
\subsection{ISO/IEC 33001}
O ISO/IEC 33001 é uma família de padrões técnicos para o desenvolvimento e manutenção de programas de computador.
\subsection{CMMI®}
O CMMI® (Capability Maturity Model ® Integration) são modelos de maturidade de software criados originalmente para o departamento de defesa dos Estados Unidos. Hoje, são usados extensivamente na indústria da tecnologia para garantir a qualidade e capacidade de diferentes aplicações. O MR-MPS-SW é totalmente compatível com o CMMI 2.0.
\subsection{Níveis de Maturidade}
O MR-MPS-SW é dividido em sete níveis de maturidade, sendo o mais baixo definido pela letra G e o mais alto pela letra A, esses níveis combinam processos de projeto e processos organizacionais, além de suas capacidades.
\subsection{Processos de Projeto}
Os processos de projeto que compõem os níveis são: Gerencia de Projetos, Engenharia de Requisitos, Projeto e Construção de Produto, Integração de Produto, e Verificação e Validação.
\subsection{Processos de Organização}
Já os processos de organização são: Gerencia de Recursos Humanos, Gerencia de Configuração, Gerencia Organizacional, Gerencia de Processos, Medição, Aquisição, e Gerência de Decisões.
\subsection{Nível B}
O nível escolhido para essa pesquisa foi nível B, segundo mais alto do MPSBR. Ele é descrito pela frase “Gerenciado Quantitativamente.” sendo composto pelos resultados evoluídos dos processos: Gerencia de Projetos, Gerencia de Processos, Gerencia Organizacional, Medição e Aquisição. Para uma empresa ser qualificada no nível B, a implementação desses processos deve ser igual ou superior à capacidade exigida pelos níveis anteriores e também à seguinte norma: Capacidade do Processo Nível B (CP-B) - O processo é previsível: um processo de capacidade CP-B deve ter a capacidade CP-E/D/C e também deve utilizar técnicas estatísticas e outras técnicas quantitativas para determinar ou prever o alcance de objetivos de qualidade e de desempenho dos processos. Ou seja, o nível B não adiciona novos processos, apenas visa evoluir, por meio de técnicas quantitativas, os resultados de processos adicionados em níveis anteriores. A definição oficial da Softex é: “Este nível de maturidade é composto pelos processos dos níveis de maturidade anteriores (G ao C). Neste nível o processo de Gerência de Projetos sofre sua segunda evolução, sendo acrescentados novos resultados para atender aos objetivos de gerenciamento quantitativo. Neste nível a implementação dos processos deve satisfazer os atributos de processo AP 1.1, AP 2.1, AP 2.2, AP 3.1 e AP 3.2 e os RAP 22 a RAP 25 do AP 4.1. A implementação dos processos selecionados para análise de desempenho deve satisfazer integralmente os atributos de processo AP 4.1 e AP 4.2. Este nível não possui processos específicos.”
\subsection{Evoluções de processo}
As evoluções de processo necessárias para a classificação B são: RAP 22, RAP 23, RAP 24 e RAP 25. Essas normas evidenciam os resultados esperados na medição de resultados dos processos. Elas incluem: Identificação das necessidades de informação dos usuários e objetivos quantitativos de qualidade e desempenho. \\
O benefício esperado do nível B é uma evolução na aplicação dos processos inclusos nos níveis mais baixos do MPSBR, é um nível de aperfeiçoamento e não diretamente de implementação.
\subsection{Benefícios}
Já os benefícios do MPSBR todo são mais claros, já que para chegar aos níveis altos, que apenas aperfeiçoam, a organização precisara passar por todos os níveis baixos que diretamente introduzem e implementam novos processos na empresa. Esses processos buscam melhorar todo o ciclo de vida de um software, desde seu planejamento e execução até a manutenção e aposentadoria dele, garantindo uma melhor utilização de recursos e um resultado de maior qualidade e confiabilidade. Por isso, clientes que precisam de softwares parrudos, como hospitais e exércitos, costumam procurar empresas qualificadas no MPSBR para desenvolver suas aplicações.
\subsection{Obstáculos}
Porém, vale-se notar que além dos benefícios, esses novos processos podem trazer problemas a uma empresa, em vários casos, por diversos motivos, as tarefas e processos extras acabam atrapalhando mais do que ajudando, por isso, antes de implementar o MPSBR em uma empresa, é importante conduzir uma pesquisa de mercado e a partir disso decidir. Nem todos conseguem se beneficiar dos métodos apresentados nesse artigo, mas aqueles que conseguem, ganham muito.

\section{Conclusão}
O MPSBR é um sistema parrudo e complexo de aprimoramento de processos para aumentar a qualidade e confiabilidade dos produtos desenvolvidos por uma empresa, porém, traz uma alta complexidade a organização. Os benefícios são claros, porém a execução e implementação é difícil. Nem todos podem se beneficiar, já que os métodos foram desenvolvidos para empresas que precisam criar produtos de qualidade que não falham.

\section{Referencias}

%\begin{itemize}
%    \item https://inbre.jabsom.hawaii.edu/wp-content/uploads/2017/05/Imaizumi-Yuko_Research-paper-2017.pdf
%    \item https://www.coloradocollege.edu/dotAsset/f543ad6e-f606-4738-97d9-7945986705b0.pdf
%    \item https://www.bates.edu/biology/files/2010/06/How-to-Write-Guide-v10-2014.pdf
%    \item https://edisciplinas.usp.br/pluginfile.php/1876409/mod_folder/content/0/HTW_Guide_Sections_3-7-2011.pdf?forcedownload=1
%    \item https://softex.br
%    \item https://blog.algartelecom.com.br/inovacao/significado-de-tics-entenda-de-uma-vez-por-todas/
%    \item https://promovesolucoes.com/quais-sao-os-niveis-de-maturidade-do-mps-br/
%    \item https://www.softex.br/wp-content/uploads/2013/07/MPS.BR_Guia_Geral_Software_2012-c-ISBN-1.pdf
%    \item https://blogdaqualidade.com.br/o-que-e-o-mps-br/
%    \item https://softex.br/mpsbr/
%    \item http://hernaski.com.br/blog/qualidade-do-produto-vs-qualidade-do-processo/1
%    \item https://zeev.it/blog/qualidade-produto-qualidade-processo/
%    \item https://pt.wikipedia.org/wiki/Associa%C3%A7%C3%A3o_para_Promo%C3%A7%C3%A3o_da_Excel%C3%AAncia_do_Software_Brasileiro
%    \item https://softex.br/mpsbr/avaliacoes/
%    \item https://cmmiinstitute.com
%    \item https://www.iso.org/standard/78526.html
%    \item https://en.wikipedia.org/wiki/ISO/IEC_33001
%    \item http://www.isdbrasil.com.br/artigos/cmmi2.0.php
%\end{itemize}

\nocite{*}
\bibliographystyle{alpha}
\bibliography{sample}


\end{document}
